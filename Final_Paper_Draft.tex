\documentclass[11pt,]{article}
\usepackage[left=1in,top=1in,right=1in,bottom=1in]{geometry}
\newcommand*{\authorfont}{\fontfamily{phv}\selectfont}
\usepackage[]{mathpazo}


  \usepackage[T1]{fontenc}
  \usepackage[utf8]{inputenc}




\usepackage{abstract}
\renewcommand{\abstractname}{}    % clear the title
\renewcommand{\absnamepos}{empty} % originally center

\renewenvironment{abstract}
 {{%
    \setlength{\leftmargin}{0mm}
    \setlength{\rightmargin}{\leftmargin}%
  }%
  \relax}
 {\endlist}

\makeatletter
\def\@maketitle{%
  \newpage
%  \null
%  \vskip 2em%
%  \begin{center}%
  \let \footnote \thanks
    {\fontsize{18}{20}\selectfont\raggedright  \setlength{\parindent}{0pt} \@title \par}%
}
%\fi
\makeatother




\setcounter{secnumdepth}{0}




\title{Statistical Analysis of Temporal and Spatial Trends in US Covid-19 Cases
and Deaths  }



\author{\Large Jason Gong and Micah Swann\vspace{0.05in} \newline\normalsize\emph{University of California, Davis}  }


\date{}

\usepackage{titlesec}

\titleformat*{\section}{\normalsize\bfseries}
\titleformat*{\subsection}{\normalsize\itshape}
\titleformat*{\subsubsection}{\normalsize\itshape}
\titleformat*{\paragraph}{\normalsize\itshape}
\titleformat*{\subparagraph}{\normalsize\itshape}


\usepackage{natbib}
\bibliographystyle{apsr}
\usepackage[strings]{underscore} % protect underscores in most circumstances



\newtheorem{hypothesis}{Hypothesis}
\usepackage{setspace}


% set default figure placement to htbp
\makeatletter
\def\fps@figure{htbp}
\makeatother

\usepackage{hyperref}

% move the hyperref stuff down here, after header-includes, to allow for - \usepackage{hyperref}

\makeatletter
\@ifpackageloaded{hyperref}{}{%
\ifxetex
  \PassOptionsToPackage{hyphens}{url}\usepackage[setpagesize=false, % page size defined by xetex
              unicode=false, % unicode breaks when used with xetex
              xetex]{hyperref}
\else
  \PassOptionsToPackage{hyphens}{url}\usepackage[draft,unicode=true]{hyperref}
\fi
}

\@ifpackageloaded{color}{
    \PassOptionsToPackage{usenames,dvipsnames}{color}
}{%
    \usepackage[usenames,dvipsnames]{color}
}
\makeatother
\hypersetup{breaklinks=true,
            bookmarks=true,
            pdfauthor={Jason Gong and Micah Swann (University of California, Davis)},
             pdfkeywords = {},  
            pdftitle={Statistical Analysis of Temporal and Spatial Trends in US Covid-19 Cases
and Deaths},
            colorlinks=true,
            citecolor=blue,
            urlcolor=blue,
            linkcolor=magenta,
            pdfborder={0 0 0}}
\urlstyle{same}  % don't use monospace font for urls

% Add an option for endnotes. -----


% add tightlist ----------
\providecommand{\tightlist}{%
\setlength{\itemsep}{0pt}\setlength{\parskip}{0pt}}

% add some other packages ----------

% \usepackage{multicol}
% This should regulate where figures float
% See: https://tex.stackexchange.com/questions/2275/keeping-tables-figures-close-to-where-they-are-mentioned
\usepackage[section]{placeins}


\begin{document}
	
% \pagenumbering{arabic}% resets `page` counter to 1 
%
% \maketitle

{% \usefont{T1}{pnc}{m}{n}
\setlength{\parindent}{0pt}
\thispagestyle{plain}
{\fontsize{18}{20}\selectfont\raggedright 
\maketitle  % title \par  

}

{
   \vskip 13.5pt\relax \normalsize\fontsize{11}{12} 
\textbf{\authorfont Jason Gong and Micah Swann} \hskip 15pt \emph{\small University of California, Davis}   

}

}








\begin{abstract}

    \hbox{\vrule height .2pt width 39.14pc}

    \vskip 8.5pt % \small 

\noindent This study provides a statistical analysis of the spatial and temporal
trends in US Covid-19 case from March 2020 - February 2021.


    \hbox{\vrule height .2pt width 39.14pc}


\end{abstract}


\vskip -8.5pt


 % removetitleabstract

\noindent  

\hypertarget{introduction}{%
\section{Introduction}\label{introduction}}

Covid-19 is a novel, highly contagious, acute respiratory virus that was
first identified in December 2019 in Wuhan, China. Over the course of
the following 14 months, this virus spread rapidly to every corner of
the globe, becoming one of the deadliest pandemics in recorded history.
In the United States, the first confirmed Covid-19 case was identified
in January 2020 and by mid-March there were confirmed cases in every
single state and North American territory. In the midst of this rapid
pandemic spread, epidemiologists and modelers struggled to accurately
forecast the spatial and temporal trends in cases and deaths. However,
with regularly updated, publicly-available covid tracking data, a
sufficient amount of data now exists to retroactively examine how cases
and deaths evolved over the course of this 14 month period. This study
utilizes the New York Times Covid Tracking Data to statistically analyze
trends in the timeseries of Covid-19 cases and deaths as well as the
spatial development of cases at the state level across the united states
using cluster analysis.

\hypertarget{methodology}{%
\section{Methodology}\label{methodology}}

\hypertarget{data-sources}{%
\subsection{Data Sources}\label{data-sources}}

Due to the fragmented nature of the US public health system, there is no
centralized governmental data repository that is updated daily with
Covid-19 case and death data. Instead, this study obtained data from the
New York Times (NYTimes) Covid-19 Tracking Project
(\url{https://github.com/nytimes/covid-19-data}). The NYTimes relies on
dozens of reporters across multiple time zones to regularly update this
tracking database with new information from press conferences, report
releases, and local databases. Datasets utilized in this analysis
reported the daily cumulative case and death counts in the US aggregated
at the national, state and county level (US.csv, US-states.csv,
US-counties.csv), respectively. Demographic data on state populations
were also obtained from the US census bureau to compare per capita
rates.

\hypertarget{data-formatting}{%
\subsection{Data Formatting}\label{data-formatting}}

\hypertarget{exploratory-data-analysis}{%
\subsection{Exploratory Data Analysis}\label{exploratory-data-analysis}}

\hypertarget{clustering-analysis}{%
\subsection{Clustering Analysis}\label{clustering-analysis}}

\hypertarget{results}{%
\section{Results}\label{results}}

\hypertarget{conclusions}{%
\section{Conclusions}\label{conclusions}}

\hypertarget{appendix}{%
\section{Appendix}\label{appendix}}





\newpage
\singlespacing 
\bibliography{master.bib}

\end{document}
